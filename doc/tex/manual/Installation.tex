%%____________________________________________________________________ 
%% File: Installation.tex
%%____________________________________________________________________ 
%%  
%% Author: Shaun ASHBY <Shaun.Ashby@cern.ch>
%% Update: 2005-11-02 17:05:05+0100
%% Revision: $Id: Installation.tex,v 1.4.2.1 2007/03/02 13:54:00 sashby Exp $ 
%%
%% Copyright: 2005 (C) Shaun ASHBY
%%
%%--------------------------------------------------------------------
\chapter{Obtaining and Installing SCRAM}
\label{ch:install}\index{SCRAM!obtaining and installing}
The procedure for creating a central or local \scram\ release is
straightforward. Normally, these steps would be performed by computing
administrators but a user can also easily install \scram\ on his/her machine.
The installation uses the standard tool\texttt{configure}, which can be run to create a Makefile.

\section{Downloading and Installing from CVS}
\index{SCRAM!downloading}

Here is a guide to installing \scram. The sources can be installed
anywhere (read access is required of course); the final installed
\scram\ script must be locatable in the user path. The prerequisite
Perl toolkit for handling templates is automatically downloaded and
installed inside the source code tree so it is no longer necessary to
support options to specifiy where it is located. This is the
installation recipe:

\begin{enumerate}
\item Create a directory \texttt{SCRAM} in which you wish to keep your
  \scram\ installation,\mbox{}
  
  \eg \texttt{mkdir SCRAM}
  
\item Change to the installation directory created above,\mbox{}
  
  \eg \texttt{cd SCRAM}
  \index{CVSROOT}  
\item Set the environment variable \texttt{CVSROOT} to point to the
  CVS repository:\mbox{}

  \scramcvsrepository
  
  Use \texttt{export} or \texttt{setenv} depending on whether your
  shell is \texttt{zsh} or \texttt{tcsh}.
  
\item Log into the CVS server as user \textit{anonymous} (password is
  \texttt{98passwd}):\mbox{}
  
  \texttt{cvs login}
  
\item Check out the version of \scram\ (for example, \thisrelease)
  required:\mbox{}
  
  \texttt{cvs co -d \thisrelease~ -r \thisrelease ~SCRAM}
  
  Once a version of \scram\ has been installed, installation 
  of any other version is achieved easily using the command\mbox{}
  
  \texttt{scram version} \textit{new-version}
  
  The final step is to run \texttt{autoconf} and \texttt{configure}
  with the required options.
  
\item Go to the \texttt{Installation} directory of the version you have
  just checked out,\mbox{}
  
  \eg \texttt{cd \thisrelease/Installation}
  
\item Run \texttt{configure}, supplying any options required for your installation.\mbox{}
  
  Several templates for the main script (\texttt{scram.pl.in}), the
  install Makefile (\texttt{Makefile.in}) and for a local site package 
  (\texttt{SCRAM\_SITE.pm.in}) are processed.
  The site-dependent settings come from \texttt{SCRAM\_SITE.pm.in} and 
  their values are passed using options to \texttt{configure}. This
  template can be modified as desired to support features needed at each site.
  
  The \texttt{configure} script supports the following options:
  
  \begin{itemize}
  \item location of the Perl executable\mbox{}\\ 
    (\texttt{--with-perlexe=LOC}).
    
  \item directory where the SCRAM database file \texttt{project.lookup}
    is located\mbox{}\\ (\texttt{--with-scramdb-dir=DIR}).
    
  \item the name of the site, 
    \texttt{SCRAM\_SITENAME} (\texttt{--with-sitename=SITE}).
    
  \item the name of the installed SCRAM script in the installation directory\mbox{}\\ 
    (\texttt{--with-install-name=NAME}).
    
  \item the directory where the SCRAM script will be located\mbox{}\\
    (\texttt{--with-install-dir=DIR}).
    
  \end{itemize} 
  A list of options can be viewed using the command \texttt{configure --help}.
\end{enumerate}

\ni Then run \texttt{make}~\texttt{install}.
The installation process will have created a script called
\texttt{scram} (the default name) in the install directory (normally \texttt{bin}).

\ni If you wish to use \scram\ directly through web page links you will
now need to configure your browser to use \scram.

\section{Configuring Netscape to Use SCRAM}\label{sec:webbootstrap}
\index{bootstrapping!using Netscape}
\index{bootstrapping!using a web browser}
This section assumes that you are using Netscape Navigator but should
also apply to most web-browsers. Follow the checklist below--

\begin{enumerate}
\item From the \textit{Edit} menu, select
  \textit{Preferences}\mylongrightarrow
  \textit{Navigator}\mylongrightarrow\textit{Helper Applications}
  
\item Click on \textit{New} in the \textit{Helper Applications} window and
  complete the form. For the \texttt{MIMEType} entry, type\mbox{}
  
  \textit{application/scram\_bootstrap}
  
\item Click on the application box to select it and enter
  
  \texttt{xterm -e scram --noreturn project -d \textsc{dir} -b \%s}
  
  where \textsc{dir} is the directory where you would like all
  downloaded projects to reside (this directory must exist already).
  
\item Click the \texttt{OK} box to save it.
\end{enumerate}

\ni Now, whenever you click on a project bootstrap file your browser
will start an \texttt{xterm} in which \scram\ will run the project
installation procedure.

\section{Controlling SCRAM Versions}
\index{SCRAM!controlling versions of}
The version of \scram\ used can be easily changed by using \scram\
itself. The current version will be displayed by typing the command

\texttt{scram version}
 
\ni which will return a string with a format like \texttt{V1\_X\_X},
where the \texttt{X}'s are digits representing the major and minor
release numbers.  To switch to a different version, say \lastrelease, the
command would be\mbox{}

\texttt{scram version \lastrelease}.

\ni If the specified version already exists in the central
installation directory, it will be used. Otherwise, \scram\ will
attempt to download the requested version from the CVS repository.
The parameters for accessing a CVS server can be made site-specific by
modifying \texttt{SCRAM\_SITE.pm.in} prior to following the
installation procedure described in the previous section.

\ni Installing a specific version will make that version the default
(unless the link name is changed to something other than \scram\ or
a different installation directory is chosen).
In any case, this version can be overridden by
setting an environment variable \texttt{SCRAM\_HOME} to point to the
top-level directory of the required version:\mbox{}

\eg \texttt{export SCRAM\_HOME=/home/user/SCRAM/\lastrelease}

\ni will force \scram\ to use version \lastrelease\ throughout the current
session.

\ni \scram\ can be made to automatically select the correct version to
use for a given project area, overridding any other version selection
mechanism. To turn this feature on, simply create the file
\texttt{scram\_version} in your project development/release
\index{\texttt{scram\_version} file}
configuration directory. The file should contain the required \scram\
version tag. Any \scram\ command issued in that area is then directed
to the correct version if it is installed. If the required version is
not installed then a warning is issued and execution of the command
continues using the current (default) version.

\section{The SCRAM Database} 
\index{SCRAM!project database}
\index{\texttt{SCRAM\_USERLOOKUPDB} variable}
\index{installing and removing projects}
\index{\texttt{scramdb} directory}
\index{\texttt{project.lookup} file}
The \scram\ database file is called \texttt{project.lookup} and 
contains lookup tables for the various projects that may be 
installed on your system. The database is read
during a \texttt{scram list} command.  Usually, the database file
lives in the \texttt{scramdb} directory of
your \scram\ installation area and is accessible to all \scram\
versions. The database directory is created automatically at
installation time. The default location can be overridden with the
environment variable \texttt{SCRAM\_USERLOOKUPDB} which should be the
full path to the \texttt{project.lookup} file.

\begin{description}
\item[Installing a project in the database]\mbox{}\\
  A project can be installed into the database using the command

\begin{scramcmd}{install}
  \inbrackets{project\_name}~\inbrackets{project\_version}
\end{scramcmd}

\ni This command is usually executed in the project area of the
project (and version) to be installed.  If the project name and
version are given then these will be used in the database in place of
the project defaults.


\item[Removing an installed project from the database]\mbox{}\\
  A project can be removed from the database using the command

\begin{scramcmd}{remove}
  \inbrackets{project\_name}~\inbrackets{project\_version}
\end{scramcmd}

\ni This command can be executed anywhere but will only work if the
user has write permissions on the \texttt{project.lookup} file.

\index{linking other SCRAM databases}
\item[Linking other databases]\mbox{}\\
  Other databases can be linked to the local installation database
  using the command

\begin{scramcmd}{db link}
  ~\inbrackets{otherdb}
\end{scramcmd}

\ni This is often useful if a local copy of a project is maintained
for development but a released project is required for access to
configuration information and libraries. All projects listed in the
linked database will be shown by \texttt{scram list}.  Note that
\texttt{scram install} and \texttt{scram remove} commands will not
affect databases that are linked in.
\end{description}

\section{Supporting Multiple Architectures}

The site \scram\ installation can be customised to support multiple
architectures (where the architecture is dependent only on the
operating system specifics). Architectures that reflect a combination of
compiler, compiler version and \texttt{OS} will be supported in later
releases and are only supported as hard-coded strings.

\ni The architecture can be determined automatically and is not limited to
the mechanism used in \texttt{V0} versions of \scram\ which use only
\texttt{uname} and required the environment variable
\texttt{SCRAM\_ARCH} to be set to change to another architecture. 
A principal check uses \texttt{Perl} internal variables to
determine the name of the current platform. This value is then
compared to an architecture map found at the end of
\texttt{SCRAM\_SITE.pm.in} which can be freely edited-- \textit{before
running the install procedure above}-- to support any
desired architecture on which \texttt{Perl} will run. 
The architecture map data looks like this:
\begin{verbatim} 
__DATA__; # Signals end of package
######################################################
#                                                    #
# These are the architecture mappings for this site  #
#                                                    #
######################################################
linux:&linux_check(2,3,4,"intel") && &is_slc_version(4) && &is_64bit():"slc4_ia64_gcc345"
linux:&linux_check(2,3,4,"amd") && &is_slc_version(4) && &is_64bit():"slc4_amd64_gcc345"
linux:&linux_check(2,3,2,"amd") && &is_slc_version(3) && &is_64bit():"slc3_amd64_gcc345"
linux:&linux_check(2,3,4,"intel") && &is_slc_version(4):"slc4_ia32_gcc345"
linux:&linux_check(2,3,2,"intel"):"slc3_ia32_gcc323"
darwin:&check_is_osx_panther():"osx103_gcc33"
darwin:&check_is_osx_tiger() && &is_ppc() :"osx104_ppc_gcc40"
darwin:&check_is_osx_tiger():"osx104_ia32_gcc40"

\end{verbatim}

\ni The first element in each line of data is the native platform. The
second element is the name of a Perl subroutine (which can be
customised to suit), found in \texttt{SCRAM\_SITE.pm}, which will 
be run and will return true or false depending on whether some criteria (\eg versions of
\texttt{glibc}, the name of an \texttt{OS} release \etc) are satisfied. If they
are, the final element will be returned as the name of the
architecture. This will be set once as soon as \scram\ is run.
Setting the environment variable \texttt{SCRAM\_ARCH} or using
\texttt{scram -arch X <CMD>} will override this value.
Combinations of subroutines can be run also as in the second example above.
Any number of architectures can be listed but the last set
of parameters that fully match will be used to give the current
architecture name.

\newpage % A new page please

%%% Local Variables:
%%% mode: latex
%%% TeX-master: "SCRAM-manual"
%%% End:

%%____________________________________________________________________ 
%% End of Installation.tex
%%____________________________________________________________________ 
%%  
