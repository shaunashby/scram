%%____________________________________________________________________ 
%% File: QuickHelpGuide.tex
%%____________________________________________________________________ 
%%  
%% Author: Shaun ASHBY <Shaun.Ashby@cern.ch>
%% Update: 2005-11-02 17:10:20+0100
%% Revision: $Id: QuickHelpGuide.tex,v 1.5 2007/02/27 16:16:01 sashby Exp $ 
%%
%% Copyright: 2005 (C) Shaun ASHBY
%%
%%--------------------------------------------------------------------
\newcommand{\cmdintro}{Command overview}
%%
\chapter{Quick Help Guide for Developers}
\label{ch:quickhelpguide}\index{help pages for developers}\index{\texttt{scram}}
This chapter is intended to be a quick help guide for developers
already familiar with working in a \scram\ project environment.  Each
of the recognised \scram\ commands is described here.
All commands provide online help too which can be accessed using

\hspace{5mm}\scram~\inbrackets{command}~\texttt{--help} 

or

\hspace{5mm}\scram~\inbrackets{command}~\texttt{-h} 

\ni For administrators setting up a project area for the first time,
refer to Chapter~\ref{ch:creatingprojects}.

\section{\scram\ version}
\index{\texttt{scram}!\texttt{version}}

\cmdintro:

\hspace{5mm}\scram~\texttt{version}~[\texttt{-c}]~[\texttt{-i}]~[\textit{version}]

\ni With no \textit{version} argument given, this command will simply show
the current \scram\ version number. If a version argument is supplied,
that version will be downloaded and installed, if not already locally available.

\begin{description}
\item[\textbf{-i}]
  Show \scram\ CVS commit info (the value of the CVS \texttt{Id} variable)
\item[\textbf{-c}]
  Print site CVS parameters to \texttt{stdout}. These parameters are used
  when downloading and installing new \scram\ versions at a site.
\end{description}


\section{\scram\ arch}
\index{\texttt{scram}!\texttt{arch}}

\cmdintro:

\hspace{5mm}\scram~\texttt{arch}

\hspace{5mm}\scram~\texttt{-arch}~\inbrackets{architecture}

\ni Print out the architecture flag for the current machine or
set the architecture to that specified.

\section{\scram\ runtime}
\index{\texttt{scram}!\texttt{runtime}}

\cmdintro:

\hspace{5mm}\scram~\texttt{runtime} [-csh,-sh \textit{or} -win]

\hspace{5mm}\scram~\texttt{runtime} [-csh,-sh \textit{or} -win]~\texttt{-file}~\textit{filename}

\hspace{5mm}\scram~\texttt{runtime} [-csh,-sh \textit{or} -win]~\texttt{-file}~\textit{filename}~\texttt{-info}~\inbrackets{variable}

\hspace{5mm}\scram~\texttt{runtime} [-csh,-sh \textit{or} -win]~\texttt{-dump}~\textit{filename}

\ni Print the runtime environment for the current development area
in \texttt{csh}, \texttt{sh} or Windows flavours. 

\example{Examples}\mbox{}

\ni Set up to include the project runtime settings
in the current \texttt{tcsh} shell environment:

\hspace{5mm}\texttt{eval}~`\scram~\texttt{runtime}~\texttt{-csh}` 

\ni Set up to include the project runtime settings
in a \texttt{bash} or Bourne shell environment:

\hspace{5mm}\texttt{eval}~`\scram~\texttt{runtime}~\texttt{-csh}` 

\ni To dump this environment to a file which can be sourced later, use

\hspace{5mm}\scram~\texttt{runtime}~\texttt{-sh}~\texttt{-dump}~\textit{env.sh}

\ni Note that from \scram\ \thisrelease\ it is no longer necessary to
set the runtime environment before using the \scram~\texttt{build}
command as this is now handled automatically. Of course, it is still
required before running any executable in a developer area.

\section{\scram\ config}

\cmdintro:

\hspace{5mm}\scram~\texttt{config}~[\texttt{-t}]~[\texttt{-f}]

\ni Dump configuration information for the current project area.

\begin{description}
\item[\textbf{-t}]
  Dump a list of configured tools, rather like "\texttt{scram tool info}", but
  in a format parseable by external scripts. This could be used to
  create RPMs or \texttt{tar} files of external products required by a project.
  
  \ni The format of each line of output is:

  \hspace{5mm} \textsf{name} : \textsf{version} : \textsf{project} : $<$\textsf{base\_path}$>$ : \textit{dependencies}
  
  \ni where the \textsf{project} entry is 0 or 1 depending on whether the
  tool refers to a \scram\ project, $<$\textsf{base\_path}$>$ is the top
  directory where the tool is installed (it can have the value
  \texttt{SYSTEM} if located in system directories, \eg /lib) and
  \textit{dependencies} lists the dependencies that the tool has on
  other external tools. This will be set to \texttt{NONE} if there are 
  no external dependencies.\\
  
\item[\textbf{-f}]
  List the tool info and project information.
\end{description}

\section{\scram\ list}
\index{\texttt{scram}!\texttt{list}}

\cmdintro:

\hspace{5mm}\scram~\texttt{list}~[\texttt{-c}]~[\texttt{-o}]~\inbrackets{projectname}

\ni List the available projects and versions installed in the
local \scram\ database (see the help for the \texttt{scram install} command).

\begin{description}
\item[\textbf{-c}]
  List the available projects and versions installed in the local
  \scram\ database without fancy formatting or header strings. The
  project name, version and installation directory are printed on \texttt{stdout}, separated
  by spaces for use in scripts.
\item[\textbf{-o}]
  Show all projects from all versions (\ie pre-\scramvx)
  of \scram\ (by default, only projects built and installed with
  \scramvx\ will be listed). This emulates the old style of output.
\end{description}

\section{\scram\ db}

\cmdintro:

\hspace{5mm}\scram~\texttt{db}~\inbrackets{subcommand} 

\ni The \scram\ database administration command. Supported subcommands are:

\begin{description}
\item[\texttt{-link}]\mbox{}\\
  Make available an additional database for project and
  list operations, \eg
  
  \hspace{15mm}\scram~\texttt{db}~\texttt{-link}~\texttt{/some/path/project.lookup}
  
\item[\texttt{-unlink}]\mbox{}\\
  Remove a database from the link list. Note this does
  not remove the database, just the link.
  
  \hspace{15mm}\scram~\texttt{db}~\texttt{-unlink}~\texttt{/some/path/project.lookup}
  
\item[\texttt{-show}]\mbox{}\\
  List the databases that are linked in.
\end{description}



\section{\scram\ urlget}

\cmdintro:

\hspace{5mm}\scram~\texttt{urlget}~\inbrackets{url}

\ni Retrieve document \texttt{URL} information. For example, show location in the cache
of a local copy of a tool description document or a requirements file
which has already been downloaded.

\section{\scram\ project}

\cmdintro:

\hspace{5mm}\scram~\texttt{project} [-t] [-d area ] [-n dir ] [-f tools.conf] projecturl [projectversion]

\hspace{5mm}\scram~\texttt{project}~\texttt{-update}~\texttt{projectversion}

\ni Set up a new project development area or update an existing one. A new area will appear in the
current working directory by default.

\ni Supported options are
\begin{description}
\item[\inbrackets{projecturl}]\mbox{}\\
  The URL of a \scram\ boot file.
\item[\inbrackets{projectversion}]\mbox{}\\ 
  Only for use with a database label.
\item[\textbf{-d}~\inbrackets{area}]\mbox{}\\
  Indicate a project installation area into which the new
  project area should appear. Default is the current working
  directory.
\item[\textbf{-n}~\inbrackets{dir}]\mbox{}\\
  Specify the name of the \scram\ development area you wish to
  create.
\end{description}

\ni Currently supported \texttt{URL} types are:
\begin{description}
\item[\textbf{database label}]\mbox{}\\	
  Labels can be assigned to installed releases of projects for easy
  access (See "scram install" command). If you specify a label you must also specify
  a project version. This command is normally used to create cloned developer areas.
\item[\textbf{-b}~\inbrackets{file}]\mbox{}\\
  A boot file on an accessible file system. This command would
  be used to create a project area from scratch on a laptop.
\end{description}

\example{Examples}\mbox{}

\hspace{5mm}\scram~\texttt{project}~\texttt{XX}~\texttt{XX\_8\_0}

\hspace{5mm}\scram~\texttt{project}~\texttt{-b}~\texttt{~/myprojects/projecta/config/boot.xml}

\ni Use the \textbf{-f} flag followed by a valid filename (which \textit{must} end in ".conf") to
allow auto setup to proceed without reading files from a repository (standalone mode).

\ni Some project template files can be obtained using the command:

\hspace{5mm}\scram~\texttt{project}~\texttt{-template}

\ni The templates will be copied to a directory called \texttt{config} in the current directory.

\ni An existing developer area for a project can be updated to a more recent version of
the \textit{same} project by running 

\hspace{5mm}\scram~\texttt{project}~\texttt{-update}~\inbrackets{VERSION} 

\ni in the developer area. If no \texttt{VERSION} is given, the command is
considered as a query and will return a list of project versions which
are compatible with the configuration of the current area.
A subsequent invocation of the command with a valid \texttt{VERSION} will then update the area
to that version.

\section{\scram\ setup}

\hspace{5mm}\scram~\texttt{setup}~[-i]~[\texttt{toolname}]~\{[\texttt{version}]~[\texttt{url}]\}~[-f \textit{tools.conf}]

\ni Allows installation/re-installation of a new tool/external package into an
already existing development area. If no toolname is specified,
the complete installation process is initiated.

\begin{description}
\item[\texttt{toolname}]\mbox{}\\
  The name of the tool to be set up.
\item[\texttt{version}]\mbox{}\\
  The version of the tool to set up.
\item[\texttt{url}]\mbox{}\\
  URL of the tool document describing the tool being set up. Supported
  URLs are \texttt{file:} and \texttt{http:}.\\
\item[\textbf{-i}]
  Turn off the automatic search mechanism allowing for more
  user interaction during setup.
\item[\textbf{-f}]
  Allow the user to specify a tools file (the
  filename \textit{must} end in ".conf"). This file contains values 
  to be used for settings of the tool.  
\end{description}


\section{\scram\ tool}

\cmdintro:

       \hspace{5mm}\scram~\texttt{tool}~\inbrackets{subcommand}

\ni Manage the tools in the current \scram\ project area.
Valid tool subcommands and arguments are:

\begin{description}
\item[\texttt{list}]\mbox{}\\ 
  List of configured tools available in the current \scram\ area.
\item[\texttt{info}~\inbrackets{tool\_name}]\mbox{}\\
  Print out information on the specified tool in the current area.
\item[\texttt{tag}~\inbrackets{tool\_name}~\inbrackets{tag\_name}]\mbox{}\\
  Print out the value of a variable (tag) for the specified tool in the
  current area configuration. If no tag name is given, then all known tag
  names are printed to \texttt{stdout}.
\item[\texttt{remove}~\inbrackets{tool\_name}]\mbox{}\\
  Remove the specified tool from the current project area.
\item[\texttt{template}~\inbrackets{TYPE}]\mbox{}\\
  Create a template tool description file of type \inbrackets{TYPE},
  where \inbrackets{TYPE} can be either "compiler" or "basic" depending on whether the
  template is for a compiler or for a basic tool. The template will be
  created in the current directory.
\end{description}

%\section{\scram\ gui}
%
%	scram gui -edit [class]
%	scram gui -show [meta type]
%
%Allow user interaction with the build Metadata.

\section{\scram\ build}

\cmdintro:

\hspace{5mm}\scram~\texttt{[--debug]}~\texttt{build}~\texttt{[options]}~\texttt{[makeopts]}~\texttt{TARGET}

\ni Run compilation in the current project area.

\textbf{--debug} can be used to turn on full \scram\ debug output.

The following long options are supported (short options can also be used):
\begin{description}
\item[\texttt{--help}]\mbox{}\\
  Show this help message.
\item[\texttt{--verbose}]\mbox{}\\            
  Verbose mode. Show cache scan progress and compilation commands
  (this will automatically set \texttt{SCRAM\_BUILDVERBOSE} to \texttt{true})
\item[\texttt{--testrun}]\mbox{}\\            
  Do everything except run \texttt{gmake}.
\item[\texttt{--reset}]\mbox{}\\              
  Reset the project caches and rescan then rebuild.
\item[\texttt{--fast}]\mbox{}\\               
  Skip checking the cache and go straight to building.
\item[\texttt{--writegraphs=$<$g,p$>$}]\mbox{}\\  
  Enable creation of dependency graphs. Set this to 'global' (g) if you
  want to create project-wide dependency graphs or 'package' (p) for
  package-level graphs. The graphs will be stored in the project working
  directory. If you set the environment variable \texttt{SCRAM\_WRITEGRAPHS=X}
  (where \texttt{X} is PS/JPEG/GIF), \scram\ will automatically create the graphs in format \texttt{X}.
  
  Note that you must have AT\&T's \texttt{dot} program installed and in
  your path to be able to use this feature.
  
\end{description}

\example{Example}\mbox{}
To refresh the current area cache, produce global dependency graphs but not run \texttt{gmake}

\hspace{5mm}\scram~\texttt{build}~\texttt{-r}~\texttt{-w=g}~\texttt{-t}

\ni Make option flags can be passed to gmake at build-time: the supported options are
\begin{description}
\item[\textbf{-n}]\mbox{}\\               
  Print the commands that would be executed but do not run them.
\item[\texttt{--printdir}]\mbox{}\\
  Print the working directory before and after entering it.
\item[\texttt{--printdb}]\mbox{}\\ 
  Print the data base of rules after scanning makefiles, then build as normal.
\item[\textbf{-j $<$n$>$}]\mbox{}\\
  The number of processes to run simultaneously.
\item[\textbf{-k}]\mbox{}\\
  Continue for as long as possible after an error.
\item[\textbf{-s}]\mbox{}\\ 
  Do not print any output.
\item[\textbf{-d}]\mbox{}\\
  Run \texttt{gmake} in debug mode.
\end{description}

\section{\scram\ install}
\index{\texttt{scram}!\texttt{install}}

\cmdintro:

\hspace{5mm}\scram~\texttt{install}~[-f]~[\inbrackets{project\_tag}~[\inbrackets{version\_tag}]] 

\ni Associates a label with the a project release listed in the \scram\ database.
This allows other users to refer to a centrally installed project by
this label rather than a remote URL reference.

\begin{description}
\item[\textbf{-f}]
  Force an installation of a project, overwriting any entries
  with the same project name and version (useful in batch processing).
\item[\inbrackets{project\_tag}]\mbox{}\\
  Override default label (the project
  name of the current release)
\item[\inbrackets{version\_tag}]\mbox{}\\
  Version tag of the current release. If version is not specified the base release version will be taken by default.
\end{description}

\section{\scram\ remove}
\index{\texttt{scram}!\texttt{remove}}

\cmdintro:

\hspace{5mm}\texttt{scram}~\texttt{remove}~[\texttt{-f}]~[\inbrackets{projectname}] [\inbrackets{projectversion}]

\ni Remove a project entry from the \scram\ project database file ("project.lookup").
\begin{description}
\item[\textbf{-f}]
  Force removal of a project, not prompting the user for confirmation (useful in batch processing).
\end{description}


%%% Local Variables:
%%% mode: latex
%%% TeX-master: "SCRAM-man"
%%% End: 

%%____________________________________________________________________ 
%% End of QuickHelpGuide.tex
%%____________________________________________________________________ 
%%  
