%%____________________________________________________________________ 
%% File: Examples.tex
%%____________________________________________________________________ 
%%  
%% Author: Shaun ASHBY <Shaun.Ashby@cern.ch>
%% Update: 2005-11-02 17:08:52+0100
%% Revision: $Id: Examples.tex,v 1.5.2.1 2007/03/02 13:54:00 sashby Exp $ 
%%
%% Copyright: 2005 (C) Shaun ASHBY
%%
%%--------------------------------------------------------------------
\chapter{Examples}\label{ch:examples}

This chapter contains some useful examples of how to work with \scram.

\section{Configuring New Projects}\label{sec:configuringprojectexample}

This section covers how to create a new project using the boot
mechanism. Normally, a boot file will instruct \scram\ to download
configuration documents from a \texttt{CVS} repository. However,
sometimes this may not be possible because the appropriately
configured files may not yet be available in a repository or it is
more desirable to make a prototype locally before committing anything.

\subsection{Creating Standalone Projects}\label{sec:configuringstandaloneprojects}

First we will describe how to boot a project from copies of
configuration files based on the templates provided by
\scram. These templates can be obtained using \texttt{scram project -t}.
The \texttt{config} directory created locally contains build
templates and a global \buildfile, plus the boot and requirements
files. The boot file looks like this:
\small{
\begin{verbatim}
<?xml version="1.0" encoding="UTF-8" standalone="yes"?>
<doc type="Configuration::BootStrapProject" version="1.0">
 <project name="TEST" version="1_0">
  <base url="file:SCRAMToolBox">
   <download url="file:/CMSconfigs" name="config/site"/>
  </base>
 
   Boot file for TEST project version 1.0.

  <config dir="config"/>
  <base url="file:config">
   <download url="file:/" name="config"/>
    <requirementsdoc name="config/requirements.xml"/>
  </base>
 </project>
</doc>
\end{verbatim}}\normalsize

\ni and the requirements document looks like this:
\small{
\begin{verbatim}
<?xml version="1.0" encoding="UTF-8" standalone="yes"?>
<doc type="BuildSystem::Requirements" version="2.0">
 <base url="file:SCRAMToolBox">
  <include url="file:/CMS/Configuration/CMSconfiguration.xml"/>
  <select name="cxxcompiler"/>
 </base>
</doc>
\end{verbatim}}\normalsize

\ni These must now be edited to suit. Here is a recipe:

\begin{description}
\item[Decide on a project name and version:]\mbox{}\\
Edit the boot file, changing the name and version of the project as
desired. In the example, the project name is \texttt{TEST} and the
version is $1.0$.
\item[Configure a toolbox with required tools:]\mbox{}\\
In this example, we configure a completely standalone toolbox as a
directory structure containing the tool descriptions needed for the
project dependencies. There are no references to \texttt{cvs:\\} type
URLS and therefore no interaction with a CVS repository.
This allows for a very fast setup process since files are
"downloaded" from a local disk.
\end{description}

\ni Now we follow a second example in which a toolbox configuration
already exists, as prepared by the software librarian, with a
published CVS tag. The boot file looks like this:
\small{
\begin{verbatim}
<?xml version="1.0" encoding="UTF-8" standalone="yes"?>
<doc type="Configuration::BootStrapProject" version="1.0">
 <project name="TEST" version="1_0">
  <base url="cvs://cmscvs.cern.ch/cvs_server/repositories/SCRAMToolBox
          ?auth=pserver&amp;user=anonymous&amp;passkey=AA_:yZZ3e&amp;version=CMS_145a_2">
   <download url="cvs:?module=SCRAMToolBox/CMSconfigs" name="config/site"/>
  </base>

  <config dir="config"/>
  <base url="file:config">
   <download url="file:/" name="config"/>
   <requirementsdoc name="config/requirements.xml"/>
  </base>
 </project>
</doc>
\end{verbatim}}\normalsize

\ni and the requirements document looks like this:
\small{
\begin{verbatim}
<?xml version="1.0" encoding="UTF-8" standalone="yes"?>
<doc type="BuildSystem::Requirements" version="2.0">
 <base url="cvs://cmscvs.cern.ch/cvs_server/repositories/SCRAMToolBox
          ?auth=pserver&amp;user=anonymous&amp;passkey=AA_:yZZ3e&amp;version=CMS_145a_2">
  <include url="cvs:?module=SCRAMToolBox/CMS/Configuration/CMSconfiguration.xml"/>
  <select name="cxxcompiler"/>
 </base>
</doc>
\end{verbatim}}\normalsize

\ni The toolbox has the CVS tag \texttt{CMS\_145a\_2} which is
formatted according to local conventions but can be any valid
(existing) tag. When the project is booted, the toolbox will be
checked out from CVS but the project configuration will be taken from
a local \texttt{config} directory and copied to the final project area.


\section{BuildSystem and Configuration Tips And Tricks}

Examples of useful tips? For configuration, could mention changes to \texttt{Self}.

\paragraph{Redefining An Existing Tool}

Currently it is not possible to have multiple versions of the same
tool. Therefore, this recipe should be followed if an installed tool
should be modified, for example to redefine the libraries used or to
add an extra runtime variable definition.

\begin{itemize}
\item Create a new template using \scram, or copy the installed version
  from \texttt{.SCRAM/architecture/InstalledTools}. If you cannot see
the \texttt{InstalledTools} directory (which sometimes happens for
%% FIXME>> Should this be ``scram set'' or will ``scram setup <tool>'' work?
developer areas), type \texttt{scram setup} or copy the tool description file
from the release area for the version of the project you're using to
the project area.

\item Edit this version of the tool file.

\item Delete the installed version by running \texttt{scram tool remove <tool>}.

\item Set up the new version of the tool, giving a \texttt{file:} URL to
  point to your new tool description file.
\end{itemize}

\paragraph{Creating an executable}

How to create a binary executable. Define real binaries in a
\texttt{bin} directory in the package.

\paragraph{Creating a test executable}

How to create a test binary executable? Binaries defined in
\texttt{test} subdirectories are automatically treated as test executables
A different template for tests (relative to normal binaries) defines custom behaviour, in this
case to move the product to a \texttt{test} storage location and defining rules for executing the test. 

\paragraph{Changing the default name of a package library}

The CMSSW example. Demonstrate how to change the name of the target,
taking the path information from the standard template for the library.

\paragraph{Creating an extra shared library}

How to use the \texttt{extra\_library\_template.tmpl} template. Types of globs
to pick up source files from a stubs directory, for example.

\paragraph{How to create a debug version of something}

Adding $<$flags \texttt{CXXFLAGS}=\texttt{"-g -O0"}$/>$ to the top-level
\buildfile\ will add the debug flags to the global compiler
flags. Alternatively, adding the statement to a package \buildfile\
will produce debug symbold in that package only (likewise for
executables and modules defined in \texttt{bin} or \texttt{test} directories.

\subsection{How To Avoid Rebuilding}

Create a private "release" area from which to create
a developer area. Use your own project database by setting
\texttt{SCRAM\_USERLOOKUPDB} to a local directory in which a
\texttt{project.lookup} is added. 
To install the project on which a release will be based, change to the
project area and type \texttt{scram install}. The project should now
appear in the list of projects returned by \texttt{scram list}.

\ni A developer area can be created using \texttt{scram project NAME
  VERSION}, where the \texttt{NAME} and \texttt{VERSION} match those
in the list of projects.

\example{Defining a New Classpath}

Following the recommended project structure conventions, here is an
example of \lbkt\texttt{classpath}\rbkt settings for a project, as
defined in the top-level project \buildfile:
\index{\texttt{classpath} tag!example}

\small{
\begin{verbatim} 
 <classpath path="+Project/+SubSystem/+Package/scripts+scripts"/>
 <classpath path="+Project/Documentation+Documentation/+doc"/>
 <classpath path="+Project/+SubSystem/+Package/src+library"/>
 <classpath path="+Project/+SubSystem/+Package/bin+binary"/>
 <classpath path="+Project/+SubSystem/+Package/test+test"/>
 <classpath path="+Project/+SubSystem/+Package/data+data_install"/>
\end{verbatim}}\normalsize

\index{structure templates! and the \texttt{classpath}}
\ni These settings will associate the top-level \texttt{src}
directory with \texttt{Project\_template.tmpl} and any directory in the next
level with the template \texttt{SubSystem\_template.tmpl}. 
The level below that will be associated with \texttt{Package\_template.tmpl}.
The last elements have a pattern-matching string before the \texttt{+}
sign: only directories matching the names \texttt{src},
\texttt{test} and \texttt{bin} at the third level below
the top-level \texttt{src} directory will be mapped to the
templates \texttt{library\_template.tmpl},
\texttt{test\_template.tmpl} and
\texttt{binary\_template.tmpl} respectively.  Thus, any
source files in a directory \texttt{src} at package-level will
source the template \texttt{library\_template.tmpl} for build
rules to build a library. Likewise, binary executables will be
built from sources located in any directory called
\texttt{bin} using a template \texttt{binary\_template.tmpl}.

\example{An Example Productstore Definition} 
To create separate \texttt{include},\texttt{bin} and \texttt{lib} subdirectories, all of which are 
architecture-dependent, the following tags would be added to the project \buildfile:
\index{\texttt{productstore} tag!example}

\small{
\begin{verbatim} 
 <productstore name="include" type="arch"/>
 <productstore name="lib" type="arch"/>
 <productstore name="bin" type="arch"/>
 <productstore name="module" type="arch" swap="t"/>
\end{verbatim}}\normalsize

\ni In the above example, the storage location for modules is architecture-dependent and is
created as \texttt{module/slc3\_ia32\_gcc323} on a system with a \scram\
architecture \texttt{slc3\_ia32\_gcc323}. The other directories
(\texttt{bin}, \texttt{lib} and \texttt{include}) are created under a
directory \texttt{slc3\_ia32\_gcc323} in the project area.

\ni To store all products on a local disk in a directory independent
of the project area, use the \texttt{path} option when defining the
product store. Suppose the local directory is called
\texttt{/localscrath/u/user}- in this case, the definition for the
product stores would look like this:

\small{
\begin{verbatim} 
 <productstore name="include" type="arch" path="/localscratch/u/user"/>
 <productstore name="lib" type="arch" path="/localscratch/u/user"/>
 <productstore name="bin" type="arch" path="/localscratch/u/user"/>
 <productstore name="module" type="arch" swap="t" path="/localscratch/u/user"/>
\end{verbatim}}\normalsize

\ni The storage directories inside the project area will be links to
the locations under \texttt{/localscratch/u/user}.

\example{Example Binary BuildFile}
Here is an example of a \buildfile\ to build two binary executables,
\texttt{mySimReaderExe} and \texttt{testObserver}:

\small{
\begin{verbatim}
<use name="Sim/Reader"/>
<bin file="testSimReader.cpp" name="mySimReaderExe"/>
<bin file="testObserver.cpp"/>
\end{verbatim}}\normalsize

\ni The object files for both binaries are linked against the
library provided by the package \texttt{Sim/Reader} to produce the executables.

\example{Example Plugin BuildFile} This example will build
a shared library which is to be used as a managed plugin:

\small{
\begin{verbatim}
<library file="SimpleUtilitiesModule.cpp" name="SimpleUtilPlugin">
 <use name="Base/Utilities"/>
 <Flags SEAL_PLUGIN_NAME="SimpleUtilPlugin"/>
</library>
\end{verbatim}}\normalsize

Another example includes more than one source file in the plugin
library:

\small{
\begin{verbatim}
<library file="*.cc,BaseUtilitiesModule.cpp" name="BaseUtilPlugin">
 <use name="Base/Utilities"/>
 <Flags SEAL_PLUGIN_NAME="BaseUtilPlugin"/>
</library>
\end{verbatim}}\normalsize


%%%%%%%%%%%%%%%%%%%%%%%%%%%%%%%%%%%%%%%%%%%%%%%%%%%%%%%%%%%%%%%%%%%%%%%%%%%%%%%%%%%%

%%% Local Variables:
%%% mode: latex
%%% TeX-master: "SCRAM-manual"
%%% End: 

%%____________________________________________________________________ 
%% End of Examples.tex
%%____________________________________________________________________ 
%%  
